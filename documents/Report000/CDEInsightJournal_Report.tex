\documentclass{InsightArticle}

\usepackage[dvips]{graphicx}
\usepackage{color}
\usepackage{listings}
\usepackage{verbatim}
\usepackage{textcomp}

\definecolor{listcomment}{rgb}{0.0,0.5,0.0}
\definecolor{listkeyword}{rgb}{0.0,0.0,0.5}
\definecolor{listnumbers}{gray}{0.65}
\definecolor{listlightgray}{gray}{0.955}
\definecolor{listwhite}{gray}{1.0}


%%%%%%%%%%%%%%%%%%%%%%%%%%%%%%%%%%%%%%%%%%%%%%%%%%%%%%%%%%%%%%%%%%
%
%  hyperref should be the last package to be loaded.
%
%%%%%%%%%%%%%%%%%%%%%%%%%%%%%%%%%%%%%%%%%%%%%%%%%%%%%%%%%%%%%%%%%%
\usepackage[dvips,
bookmarks,
bookmarksopen,
backref,
colorlinks,linkcolor={blue},citecolor={blue},urlcolor={blue},
]{hyperref}


\title{CDE Applied to Executable Papers}

%
% NOTE: This is the last number of the "handle" URL that
% The Insight Journal assigns to your paper as part of the
% submission process. Please replace the number "1338" with
% the actual handle number that you get assigned.
%
\newcommand{\IJhandlerIDnumber}{3261}

\lstset{frame = tb,
       framerule = 0.25pt,
       float,
       fontadjust,
       backgroundcolor={\color{listlightgray}},
       basicstyle = {\ttfamily\footnotesize},
       keywordstyle = {\ttfamily\color{listkeyword}\textbf},
       identifierstyle = {\ttfamily},
       commentstyle = {\ttfamily\color{listcomment}\textit},
       stringstyle = {\ttfamily},
       showstringspaces = false,
       showtabs = false,
       numbers = left,
       numbersep = 6pt,
       numberstyle={\ttfamily\color{listnumbers}},
       tabsize = 2,
       language=[ANSI]C++,
       floatplacement=!h
       }

\release{1.10}

\author{Philip Guo$^{1}$, Luis Ibanez$^{2}$}
\authoraddress{$^{1}$Stanford University\\
$^{2}$ Kitware Inc.}

\begin{document}


%
% Add hyperlink to the web location and license of the paper.
% The argument of this command is the handler identifier given
% by the Insight Journal to this paper.
%
\IJhandlefooter{\IJhandlerIDnumber}


\ifpdf
\else
   %
   % Commands for including Graphics when using latex
   %
   \DeclareGraphicsExtensions{.eps,.jpg,.gif,.tiff,.bmp,.png}
   \DeclareGraphicsRule{.jpg}{eps}{.jpg.bb}{`convert #1 eps:-}
   \DeclareGraphicsRule{.gif}{eps}{.gif.bb}{`convert #1 eps:-}
   \DeclareGraphicsRule{.tiff}{eps}{.tiff.bb}{`convert #1 eps:-}
   \DeclareGraphicsRule{.bmp}{eps}{.bmp.bb}{`convert #1 eps:-}
   \DeclareGraphicsRule{.png}{eps}{.png.bb}{`convert #1 eps:-}
\fi


\maketitle


\ifhtml
\chapter*{Front Matter\label{front}}
\fi


\begin{abstract}
\noindent
This paper describes how to use the CDE tool for packaging a reproducible paper
as a binary blob suitable to be rexecuted in a destination machine.

This paper is intended to be a tutorial on how to prepare binary submissions to
the Insight Journal. This adheres to the fundamental principle that scientific
publications must facilitate \textbf{reproducibility} of the reported results.
\end{abstract}

\tableofcontents

\section{Introduction}

Since the Inception of the Insight Journal in 2005 we have been expecting
papers to include source code, data and parameters as an integral part of a
reproducible paper. This combination work very well in the context of ITK
and CMake, but becomes challenging when more tools are introduced. For example,
when additional tools and libraries such as MPI, FFTW and MINC are required as
part of the build process.

It has become necessary to explore more general mechanisms for packaging the
computational environment of the author in such a way that it can be used by
the reader of a reproducible paper.

Here we illustrate the use of CDE, a tool developed by Philip Guo at Stanford
University, as a packager of binary blobs the can be transferred to another
computer to be executed there.

\section{About CDE}

CDE is a

The CDE web site provides excellent documentation on the tool itself, so we
will focus here on illustrating how to use it, through well defined steps, for
the purpose of packaging the accompanying material of a reproducible paper.

\section{Preparing a Reproducible Paper}

\section{Submitting a Reproducible Paper}

\section{Reviewing a Reproducible Paper}

\section{Reading a Reproducible Paper}

\section{Running in a Virtual Machine}

\section{Running in The Cloud}

%%%%%%%%%%%%%%%%%%%%%%%%%%%%%%%%%%%%%%%%%
%
%  Insert the bibliography using BibTeX
%
%%%%%%%%%%%%%%%%%%%%%%%%%%%%%%%%%%%%%%%%%

\bibliographystyle{plain}
\bibliography{InsightJournal}


\end{document}
